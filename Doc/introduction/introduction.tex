\chapter{Einleitung}

MicroCore ist eine CPU mit Harvard-Architektur, die eine Teilmenge der Forth-Sprache als Maschinensprache benutzt. MicroCore wurde in Zusammenarbeit mit der Firma Send GmbH in Hamburg entwickelt. Für die Version 1.71 wurde ein C-Compiler mit SCC (Stack-C-Compiler) implementiert. Ziel dieser Arbeit ist das Entwickeln einer auf Eclipse basierten Entwicklungsumgebung, die den SCC als Compiler verwendet. Es sollte möglich sein, den ganzen Prozess, vom Quellcode bis zur Kompilierung und dem Herunterladen auf das HW-Target in der Entwicklungsumgebung durchführen zu können. Im ersten Kapitel "`\nameref{chap:platform}"' wird beschrieben, welche Möglichkeiten Eclipse als Plattform bietet, um eine Entwicklungsumgebung zu entwickeln. Im Kapitel "`\nameref{chap:compilerintegration}"' wird erklärt, wie der Compiler integriert wurde. Im dritten Kapitel "`\nameref{chap:programlaunch}"' wird beschrieben, wie das kompilierte uForth Programm aus der Entwicklungsumgebung gestartet werden kann und was dafür implementiert wurde. Im Kapitel "`\nameref{chap:forthcommunication}"' wird beschrieben, wie die Kommunikation mit dem Forth-Prozess, entworfen, implementiert und getestet wurde. Im nächsten Kapitel "`\nameref{chap:fortheditor}"' wird beschrieben, wie Xtext dazu verwendet wurde, um einen uForth Editor in der Entwicklungsumgebung zu integrieren. Im Kapitel "`\nameref{chap:debugger}"' wird gezeigt, auf welche Arten der Debugger in die Entwicklungsumgebung eingebunden wurde und wie diese noch erweitert werden könnten. Im nächsten Kapitel "`\nameref{chap:settings}"' werden die Einstellungen, die in der Entwicklungsumgebung vorgenommen werden können, beschrieben. Im Kapitel "`\nameref{chap:optimizer}"' werden einige Peephole Optimierungen für den Compiler vorgestellt und mit der aktuellen Peephole-Optimierung des Forth-Cross-Compilers verglichen.
