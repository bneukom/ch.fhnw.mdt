\chapter{Forth Kommunikation}
\label{forthcommunication}

In diesem Kapitel wird beschrieben, wie die Entwicklungsumgebung mit dem Forth Prozess kommuniziert. Die Kommunikation mit dem Forth Prozess ist von zentraler Bedeutung, da viel der Funktionalität der Entwicklungsumgebung davon abhängt, dass die Kommunikation stabil läuft. Es wird gezeigt wie die Kommunikation designt, implementiert und getestet wurde.

\section{Prozess Kommunikation}
Um mit dem Prozess zu kommunizieren gibt es einige Alternativen welche ich aufzeigen möchte.

\subsection{GDB/MI-Commands}

Eine Möglichkeit mit dem Prozess zu kommunizieren wäre, ein MaschineInterface (MI) wie es für den GDB implementiert wurde, zu gebrauchen. GDB/MI ist ein Linien basiertes Maschinen orientiertes Text Interface zu dem GDB. Es wurde dazu entwickelt, damit der GDB als Debugger in ein grössers System einfacher einbindbar ist.\cite{gdb} Eine MI ähnliches Interface wäre mit grossem Aufwand verbunden, da das ganze Interface definiert werden müsste. Dafür könnte aber viel des CDT Debugging Mechanismus verwendet werden, da das CDT Debugging auch auf MI basiert. Auch ist der CDT Debugger sehr kompliziert und es wäre ein grosser Einarbeitungsaufwand notwendig um den Forth Debugger darauf basieren zu können.\cite{mieclipse}

\subsection{Direkte Kommunikation mit dem Prozess}

Eine weitere Möglichkeit wäre, direkt die Befehle an den Prozess senden und auf Antworten warten. Dafür müsste einige Klassen implementiert werden, um das Kommunizieren zu vereinheitlichen und vereinfachen.

\subsubsection{Probleme bei der Kommunikation}

Bei der Kommunikation mit dem Prozess können einige Probleme Auftreten. Es kann sein, dass der Prozess plötzlich keine Antworten mehr gibt. Auch weiss man nicht, wie lange es dauern wird, bis der Prozess Antwort gibt. Diese Probleme müssen mit den im vorigen genannten Klassen sauber gelöst werden.

\section{API Design}

In einem ersten Schritt wurde ein API designt, welches verwendet werden soll um die Kommunikation mit dem Prozess möglichst einfach zu halten.

TODO API

\section{Implementierungs Details}

TODO class diagramms

\subsection{Kommunikation mittels Commands}

\subsection{Forth Output Parsing}

\subsection{Await auf Resultate}

