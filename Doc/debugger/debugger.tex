\chapter{Debugger}

In diesem Kapitel wird beschrieben, wie der Debugger in Eclipse integriert wurde. Es wird aufgezeigt, was für Möglichkeiten existieren einen Debugger in Eclipse zu integrieren und welche implementiert wurden.

\section{Breakpoints}
Als erstes müssen für den Debugger Breakpoints für Forth gesetzt werden können. Dafür stehen zwei Möglichkeiten zur Verfügung auf welche ich kurz eingehen möchte.

\subsection{Per Konsole}

Die erste Möglichkeit ist, die Breakpoints per Konsole zu setzen. Das senden der Commands funktioniert mit den im Kapitel \ref{forthcommunication} beschriebenen Klassen. In der Konsole kann der Command

%
\begin{verbatim}
debug _function
\end{verbatim}
%
abgesetzt werden, um einen Breakpoint zu setzen.

TODO Konsolen Screenshot

\subsection{Im Source Code}

Eine weitere Möglichkeit ist, Breakpoints im C-File zu setzen. Dies wurde so umgesetzt, dass der Breakpoint nur auf eine Funktionsdefinition gesetzt werden kann. Alle anderen Zeilen des C-Source Codes können nicht direkt auf den übersetzten Forth Code abgebildet werden und sind deshalb nicht erlaubt für die Breakpoints.
\newline
Eclipse CDT stellt den Abstract Synatx Tree (AST) des C-Files zur Verfügung. Mit Hilfe des AST kann überprüft werden, ob sich der Breakpoint wirklich auf einer Funktionsdefinition befindet.

TODO Screenshot

\section{Konsolen basierter Debugger}

Eine erste Implementation des Debuggers ist, den schon existierenden Forth Konsolen Debugger im Eclipse zu integrieren. Dieser kann mit dem im Kapitel \ref{forthcommunication} beschriebenen Prozess Kommunikationsmitteln angesteuert und in einer Eclipse Console View angezeigt werden.

TODO Screenshot

\section{Forth Debugger}

Frontend für Konsolen Debugger.

\subsection{Neue Debugger Aktionen}

\subsubsection{Jump Action}

\subsubsection{Over Action}

\subsection{Stack View}

\subsubsection{User Interface}

\subsection{Memory View}

\subsubsection{User Interface}

\section{C Debugger}