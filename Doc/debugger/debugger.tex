\chapter{Debugger}

In diesem Kapitel wird beschrieben, wie der Debugger in Eclipse integriert wurde. Es wird aufgezeigt, was für Möglichkeiten existieren einen Debugger in Eclipse zu integrieren und welche implementiert wurden.

\section{Breakpoints}
Als erstes müssen für den Debugger Breakpoints für Forth gesetzt werden können. Dafür stehen zwei Möglichkeiten zur Verfügung auf welche ich kurz eingehen möchte.

\subsection{Per Konsole}

TODO Konsolen Screenshot

\subsection{Im Source Code}

TODO Screenshot

\section{Konsolen basierter Debugger}

Eine erste Implementation des Debuggers ist, den schon existierenden Forth Konsolen Debugger im Eclipse zu integrieren. Dieser kann mit dem im Kapitel \ref{forthcommunication} beschriebenen Prozess Kommunikationsmitteln angesteuert und in einer Eclipse Console View angezeigt werden.

TODO Screenshot

\section{Forth Debugger}

Frontend für Konsolen Debugger.

\subsection{Neue Debugger Aktionen}

\subsubsection{Jump Action}

\subsubsection{Over Action}

\subsection{Stack View}

\subsubsection{User Interface}

\subsection{Memory View}

\subsubsection{User Interface}

\section{C Debugger}